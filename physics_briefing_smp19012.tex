\documentclass[12pt]{article}
\usepackage{graphicx}

\newcommand{\WZ}{\ensuremath{\mathrm{W}\mathrm{Z}}}
\newcommand{\WW}{\ensuremath{\mathrm{W}^\pm\mathrm{W}^\pm}}
\newcommand{\jet}{\ensuremath{\mathrm{j}}}
\newcommand{\mjj}{\ensuremath{m_{\jet\jet}}}
\begin{document}

\title{Probing the Standard Model with a boson-boson collider}
\date{\today}
\maketitle

Can we build a vector boson collider in the tunnel of the Large Hadron Collider (LHC)? At first glance this is impossible as the heavy gauge bosons have extremely short lifetimes. The W and Z bosons with energies of 1 TeV travel only distances of about $0.1 \times 10^{-15}$ meters before decaying to other particles and can't be used as targets or projectiles; or can they? During rare occasions a W or Z boson is created inside a proton and even more rarely this happens inside two colliding protons at the LHC resulting in unlikely collisions of two bosons. There are approximately 1 billion proton-proton collisions per second at the LHC and these extremely rare vector boson scattering (VBS) events are recorded once every 1000 billion proton-proton collisions.

\begin{figure*}[htb]
\centering
\includegraphics[width=0.30\textwidth]{figures/leptonic_vbs_triple.pdf}
\includegraphics[width=0.30\textwidth]{figures/leptonic_vbs_quartic.pdf}
\includegraphics[width=0.30\textwidth]{figures/leptonic_vbs_higgs_tchannel.pdf}
\caption{Illustrative Feynman diagrams of a VBS process contributing to the EW-induced production of events containing $\WW$ boson pairs decaying to leptons and two forward jets. The diagrams with the triple-gauge-coupling vertex (left), quartic-gauge-coupling vertex (center), the t-channel Higgs boson exchange (right) are shown.}
\label{fig:feynman}
\end{figure*}

The VBS process provides a treasure trove of information on the structure of the Standard Model (SM) of particles as it includes contributions from triple-gauge-coupling, quartic-gauge-coupling, and Higgs boson vertices as illustrated in Figure~\ref{fig:feynman}. In fact, the LHC was built with a guaranteed discovery: we would either find a Higgs boson or discover new physics in VBS at high energies. This is due to the fact that if we calculate the amplitude of the triple-gauge-boson interaction (Figure~\ref{fig:feynman}, left)  we find that the amplitude grows with center of mass energy (E) with $E^4$ behavior. This means that the theory runs into trouble at high energies as the probability can not be bigger than 1. The inclusion of the quartic-gauge-coupling diagram (Figure~\ref{fig:feynman}, center) cancels the undesired $E^4$ behavior but a part behaving as $E^2$ remains. Only after the inclusion of the Higgs boson exchange diagram (Figure~\ref{fig:feynman}, right) a complete cancelation is achieved with a desired behavior of the scattering amplitude. Thus, studies of the VBS processes provide key insight to the Higgs sector and to the quartic gauge couplings. 

Using the data collected during 2016-18 CMS has performed detailed studies of the production of the same-sign $\WW$ and $\WZ$ boson pairs using the leptonic decays (electrons and muons). The VBS final states at the LHC have a striking signature  with two jets (originating from the two initial quarks that initially radiated the W or Z bosons) in the opposite hemispheres of the CMS detector closer to the LHC beamline and with a large invariant mass ($\mjj$) as shown in Figures xx and yy (event displays). The $\WW$ and $\WZ$ production modes are studied together by simultaneously measuring them using several kinematic observables.  

The CMS collaboration had reported the first observation of the EW  $\WW$ production at 13 TeV with significance greater than 5 standard deviation using the data collected in 2016. Results close to 5.0 standard deviations indicate a one in a million chance of the result arising from a statistical fluctuation.  The larger details allows to make the first detailed studies of the $\WW$ production as functions of different kinematic variables. An excellent agreement with the SM predictions are reported as can be seen in Figure~\ref{fig:signal} (left). Machine learning is used to establish the first observation of the electroweak production of the $\WZ$ boson pairs with the CMS detector as shown in Figure~\ref{fig:signal} (right).  The observed statistical significance is 6.8 standard deviations with 5.3 standard deviations expected from the SM predictions. Strong constraints on anomalous quartic-gauge-couplings are also set using the $\WW$ and $\WZ$ final states. 

\begin{figure*}[htb]
\centering
\includegraphics[width=0.49\textwidth]{figures/ssww_wzsel_bdt_2019.pdf}
\includegraphics[width=0.49\textwidth]{figures/ssww_wwsel_mjj_2019.pdf}
\caption{Distributions of $\mjj$ in the $\WW$ signal region (left) and  BDT score in the $\WZ$ signal region. The bottom panel in each figure
shows the ratio of the number of events observed in data to that of the total SM prediction.
The gray bands represent the uncertainties from the predicted yields.}
\label{fig:signal}
\end{figure*}

This first observations of the  electroweak production of $\WW$ and $\WZ$  boson pairs are important milestones towards precision tests of VBS at the LHC, and there is much more to be learned from the LHC Run 3 data. Models of physics beyond the SM predict enhancements to VBS via modifications to the Higgs sector, or from the presence of additional resonances. Studies demonstrate that the High Luminosity LHC, due to enter operation in the middle 2020s, should even allow a direct investigation of longitudinal W-boson scattering with great sensitivity to the Higgs mechanism.

\end{document}

